\section{Background}
\subsection{Relationships and Urban Space}

Our study explores the role of POIs in relationship-building, their activity, and social capital, emphasizing the importance of different third places for different communities based on proximity, affordability, and culture\cite{b3}. We draw upon the findings of Mazumdar et al. (2018), who identified POIs, walkability, and land use as the most critical built environmental factors for relationship-building. By analyzing a large, dynamic dataset of online reviews from the Yelp Open Dataset, we show how POIs can provide opportunities for individuals to form new social relationships or revitalize existing ones. Our work also extends classic studies on the built environment's impact on social interaction by responding to theories that certain individuals are excluded from certain destinations due to inaccessibility, exclusivity, or cost. Ultimately, our findings suggest that POIs are a valuable resource for understanding how a city or type of place serves its residents and visitors and can support specific types of social relationships.

\subsection{LBSN and POI Data Analysis}
The field of urban computing has developed numerous methods for gathering and organizing POI data from various sources and contributors\cite{b4}. This research has highlighted the potential for POIs to enhance user experience in recommendation systems, especially in mobile applications. Our analysis utilizes a location-based social network to explore the relationship between place and space through user-generated text content. Prior analyses of LBSN data have used geolocated social media to reveal the emotional state of different areas in a city. POI data from sources such as Twitter, Foursquare, Yelp, and Facebook has been used to examine how culture and weather impact when POIs are open, how their patronage varies over time, and to recommend new POIs to users. The semantics of POI labels have also been used to show how similar POIs cluster geographically and how neighborhoods change over time. Our research builds on this work by disambiguating between different types of social ties (e.g. family, friend) to better understand how people access amenities in different relationships.

\subsection{Urban Pulse}
The concept of urban pulse summarizes the spatio-temporal activity in a city by combining techniques from computational topology with visual analytics to identify, explore and analyze pulses across cities. This is accomplished by modeling the urban data as a collection of time-varying scalar functions over different temporal resolutions, where the scalar function represents the distribution of the corresponding activity over the city. The topology of this collection is then used to identify the locations of prominent pulses in a city, and the pulse is characterized as a set of beats which capture the level of activity at that location over different time steps and resolutions. The urban pulse framework includes a visual interface that can be used to explore pulses within and across multiple cities, and to also compare multiple settings.

\subsection{Scalar functions and Critical Points Determination}
In this project, we focus on modeling data as scalar functions over time, with a particular interest in the spatial region corresponding to a city, represented by a planar domain. The scalar function used is the density function, which captures the level of activity over different locations in a city based on input data points of Yelp Open DataSet. We used the same techniques mentioned in the paper\cite{b1}. The topological features of the scalar function are efficiently computed using a piecewise linear function defined on a triangular mesh, with the function computed for each discrete time step. Critical points of the function are classified based on the behavior of the function within a local neighborhood, and their topological persistence is used to identify locations of interest. The effectiveness of topological persistence as an importance measure has been demonstrated in various applications.