\section{Introduction}
Urban areas are complex and dynamic environments, constantly changing and evolving over time. As cities grow and change, they present a range of challenges for city planners, architects, and human behavioral experts to understand and navigate. Recent advancements in technology have enabled the collection and analysis of vast amounts of data, providing new opportunities to gain insights into the dynamics of urban environments. 
This paper builds on two previous studies, one focused on the concept of "urban pulses\cite{b1}" and the other on the use of points of interest (POIs) to understand the social fabrics\cite{b2} of urban areas. The concept of urban pulses captures the spatio-temporal activity in a city across multiple temporal resolutions, while POIs provide a rich source of data on how social ties are supported by local amenities. In recent years, with the emergence of online crowdsourcing review sites, such as Yelp, it has become increasingly common for people to share their experiences at POIs and how they interact socially within those POIs. This provides an opportunity to investigate how social interactions occur in urban areas, and how POIs play a role in these interactions. In this paper, we aim to examine the relationship between POIs and social interactions in Indianapolis city.

To achieve our goal, we use computational text mining techniques to analyze user reviews from the Yelp Open Dataset. We extract reviews that mention relationship keywords, such as "friend", "family", "romantic", and "professional" relationships, and classify them into their respective relationship types. We also identify the POIs and POI categories where these relationships are mentioned. In addition, we apply urban pulse analysis to check the activity across multiple temporal resolutions. Our analysis shows that different parts of the city host different types of relationships. For instance, reviews referencing romantic relationships are more likely to occur in downtown areas, while reviews mentioning children are more dispersed in suburban areas. We also find that certain POIs support different types of relationships more than others. For example, restaurants are more likely to be associated with social interactions compared to other POIs. Our study highlights the potential of using online crowdsourcing review sites as a rich source of information about social relationships and their spatial patterns in urban areas. In addition to spatial analysis, we use urban pulses to capture the spatio-temporal activity in the city. We utilized Urban Pulses to analyze and compare different pulses in Indianapolis city. We demonstrate the utility of our approach by exploring the relationship between POIs and social interactions within the city, but we acknowledge that there are limitations to our study, such as the use of only one dataset and the focus on a single city.

The primary contribution of this paper is the integration of urban pulse analysis with POI data to gain a more comprehensive understanding of the social fabrics of urban areas. This approach provides a novel perspective on urban environments that can inform city planning and design decisions. Urban planners can use these findings to reflect upon what kinds of places help support ties, which ties may need more places for their outings, and how a city can evaluate whether its social infrastructure supply is meeting the demands of residents and visitors.
