\section{Limiations and Future Work}

Despite the insights provided by our study, there are several limitations that must be acknowledged. First, our reliance on user-generated content may result in inaccurate reflections of actual social ties and relationships. Additionally, the keywords we used may not have been sufficient to capture the wide variety of relationships that use POIs. Our analysis was also limited to English-oriented keywords, which may have missed out on relationship words that are not in English, such as abuela for "grandmother" in Spanish. Furthermore, the term "partner" is ambiguous as it could refer to a professional or romantic relationship, which could lead to misinterpretation of results. Lastly, our analysis only included a limited number of cities, which may not be representative of other regions.

Further research and analysis could be conducted to explore the spatio-temporal activity and social ties of a region, which could provide valuable insights. Future studies should also explore more nuanced and inclusive approaches to dealing with relationship words. They should also consider using keywords in other languages to ensure more comprehensive analyses. Additionally, studies could explore the use of location-based social networks to suggest POIs based on who the user wants to spend time with. Disaggregating restaurant/food categories into subtypes may also be helpful in identifying stronger connections between relationship types and locales. Finally, future work could focus on addressing the issue of reinforcing or "branding" particular spaces as suitable or unsuitable for certain activities. In terms of technical limitations, the spatial analysis may not have considered the density of POIs that do not have relationship keywords. Further work is needed to disaggregate the restaurant/food categories into subtypes, such as upscale dining, sports bars, hot pot restaurants, etc. which may be used by pairs and groups for different purposes. Lastly, users should be cautioned that these results only present one perspective of cities and social network relationships, and a lack of POIs or information about a place does not imply that a place is not suitable for social life or activities (and vice versa).

\vspace{12pt}


\begin{thebibliography}{00}
\bibitem{b1} Miranda, F., Doraiswamy, H., Lage, M., Zhao, K., Gonçalves, B., Wilson, L., Hsieh, M., \& Silva, C. T. (2016). Urban Pulse: Capturing the Rhythm of Cities. ArXiv. https://doi.org/10.1109/TVCG.2016.2598585
\bibitem{b2} Bendeck, Alexander, and Clio Andris. "Text Mining and Spatial Analysis of Yelp Data to Support Socially Vibrant Cities." (2022).
\bibitem{b3} Sharon Zukin. 1998. Urban lifestyles: diversity and standardisation in spaces of
consumption. Urban Studies 35, 5/6 (1998), 825–839.
\bibitem{b4} Jennings Anderson, Dipto Sarkar, and Leysia Palen. 2019. Corporate editors
in the evolving landscape of OpenStreetMap. ISPRS International Journal of
Geo-Information 8, 5 (2019), 232.
\bibitem{b5}Yelp, Inc. 2022. Fast Facts. https://www.yelp-press.com/company/fast-facts/
default.aspx
\bibitem{b6} Yelp, Inc. 2022. Yelp Open Dataset: An All-Purpose Dataset for Learning. https:
//www.yelp.com/dataset
\bibitem{b7} U.S. Bureau of Labor Statistics. 2020. American Time Use Survey (ATUS). https:
//www.bls.gov/tus/database.html
\bibitem{b8} Kim, S., \& Kim, S. (2018). Social capital and urban public transportation use: Evidence from the United States. Journal of Transport Geography, 71, 142-150.
\bibitem{b9} Yelp, Inc. 2022. All Category List - Yelp Fusion. https://www.yelp.com/developers/
documentation/v3/all\_category\_list
\end{thebibliography}