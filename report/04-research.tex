\section{Results}

\begin{table}[htbp]
  \renewcommand{\arraystretch}{1.2}
  \caption{Occurrence of Relationship Words in Reviews per 1,000}
  \label{table:poi_stats}
  \centering
  \begin{tabular}{|p{1.3cm}|p{1cm}|p{1cm}|p{1cm}|p{1cm}|p{1cm}|}
    \hline
    \textbf{Metro} & \textbf{Words} & \textbf{Family} & \textbf{Rom.} & \textbf{Friends} & \textbf{Profes.} \\
    \hline
    Philadelphia & 167.2 & 94.5 & 17.8 & 41.2 & 13.7 \\
    \hline
    Tucson & 142.3 & 82.1 & 14.6 & 29.8 & 15.8 \\
    \hline
    Tampa & 156.8 & 85.6 & 21.3 & 34.5 & 15.4  \\
    \hline
    Indianapolis & 153.5 & 93.7 & 19.1 & 31.2 & 9.5 \\
    \hline
    Nashville & 169.1 & 100.2 & 15.9 & 40.5 & 12.5 \\
    \hline
  \end{tabular}
\end{table}

Table 4 shows the occurrences of various relationship keywords in the reviews of five cities, namely Philadelphia, Tucson, Tampa, Indianapolis, and Nashville. The most frequent relationship keyword across these cities is "family", with an average of 91.2 occurrences per 1,000 reviews. This is followed by "friends" (average of 17.7 per 1,000 reviews), "romantic" (average of 17.7 per 1,000 reviews), "professional" (average of 11 per 1,000 reviews). It is interesting to note that the occurrences of these relationship keywords vary across the cities, with Nashville having the highest average occurrences of relationship keywords (169.1 per 1,000 reviews) and Tucson having the lowest (142.3 per 1,000 reviews). These differences may reflect the cultural, social, and economic characteristics of these cities.

Spatial analysis is a powerful tool that allows us to investigate how the physical environment influences human behavior and social interactions. By examining the spatial distribution of certain social ties in different cities and neighborhoods, we can gain valuable insights into the factors that promote or hinder the formation of these ties. For example, research has shown that urban areas with high levels of walkability and public transit use tend to promote social ties by creating more opportunities for face-to-face interaction\cite{b8} and reducing reliance on private vehicles. Similarly, neighborhoods with strong community organizations and social networks may provide a supportive environment for the formation of close-knit relationships such as friendships and family ties. Further research is needed to explore the complex relationship between spatial patterns and social ties, and to identify ways to promote more equitable and supportive social environments. By leveraging the power of spatial analysis, we can gain a deeper understanding of the social dynamics of our cities and communities, and work towards building more inclusive and connected societies.